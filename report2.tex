%\documentclass[12pt]{report}
%\documentclass[12pt]{scrartcl}
\documentclass[12pt]{article}

\usepackage[portuguese]{babel}
\usepackage[utf8]{inputenc}
\usepackage{parskip}
\usepackage{indentfirst}
\usepackage{mathtools}
\usepackage{amsmath, amsthm, amssymb, amsfonts}
\usepackage[mathletters]{ucs}
\usepackage{algorithm}
\usepackage{algpseudocode}
\usepackage{tikz}
\usetikzlibrary{matrix}
\usepackage{graphicx}
\usepackage{caption}
\usepackage{subcaption}

\usepackage[margin=3cm]{geometry}
\usepackage{graphicx}
\setlength{\parindent}{0.5cm}
\usepackage{listings}
%\lstset{basicstyle=\footnotesize, keywordstyle=\color{blue}, language=C}  
\lstset{language=C,
                basicstyle=\footnotesize\ttfamily,
                keywordstyle=\color{blue}\ttfamily,
                stringstyle=\color{red}\ttfamily,
                commentstyle=\color{green}\ttfamily,
                morecomment=[l][\color{magenta}]{\#} }
                
\title{ASA - Relatório 2}
%\subtitle{Análise e Síntese de Algoritmos}

\author{Marisa Roque - 76653}
\date{\today}

\begin{document}
\maketitle


\section*{Introdução}

Foi apresentado um problema que consistia em modelar um sistema para o controlo de multidões dado um mapa de uma cidade com pontos onde as pessoas se podem concentrar e ligações entre esses pontos. Este modelo tem que garantir a não existência de comunicação entre um conjunto de pontos críticos. Para solucionar e para garantir a segurança da cidade as forças de segurança podem barrar algumas da ligações, de modo a evitar que seja iniciado um motim, minimizando os custos desta operação.


\section*{Descrição da solução}

Este problema pode ser modelado com um grafo, em que os pontos onde as pessoas se podem concentrar serão os vértices do grafo e as estradas entre estes pontos serão os arcos do grafo. O problema exposto desta maneira dá origem a um grafo não dirigido. 

A passagem de pessoas entre os vários pontos da cidade constitui um fluxo de pessoas, o que em teoria de grafos equivale a um problema de fluxos. No entanto, este requer que os arcos entre os vértices possuam uma capacidade especificada e que o grafo seja dirigido. O segundo problema é resolvido criando um par de arcos anti-paralelos para cada ligação entre pontos. 

Para o primeiro problema, como não é especificado a capacidade de cada estrada, assume-se que todas terão uma capacidade equivalente. Como a capacidade num problema de fluxo de grafos é meramente relativa, escolheu-se o valor unitário para todos os arcos.

O objectivo deste projecto é descobrir o valor mínimo de estradas a cortar pelas forças de segurança, ou seja descobrir o valor de corte mínimo de um grafo. Na teoria de grafos, o corte mínimo de um grafo é equivalente ao fluxo máximo do mesmo, pelo Teorema do Fluxo Máximo e Corte Mínimo \cite{lamport94}.

O algoritmo clássico para determinar o fluxo máximo de um grafo, entre um vértice de origem $s$ e um vértice de destino $t$, é o método de Ford-Fulkerson. Este método considera duas ideias importantes em problemas de fluxo: redes residuais e caminhos de aumento. 
Com a condição inicial de um fluxo nulo, o método irá repetir enquanto existir um caminho de aumento na rede residual entre $s$ e $t$.  Em cada iteracção, o fluxo ao longo desse caminho é aumentado com o mínimo das capacidades da rede residual ao longo do mesmo. 

O método usado neste projecto é o Edmonds–Karp, que é uma variante do método Ford-Fulkerson, em que os caminhos de aumento são escolhidos usando um método de pesquisa em largura (BFS) na rede residual. A pesquisa dos caminhos de aumento desta forma garante que o caminho encontrado seja sempre o caminho mais curto, que por sua vez garante uma ordem de grandeza do tempo de execução do algoritmo de $O(VE^2)$, sendo $V$ os vértices e $E$ os arcos.


A solução para o problema apresentado foi implementada em ANSI C, aproveitando algum código de representação de grafos do projecto anterior. 

Na solução apresentada, a representação do grafo foi efectuada com um vector de listas ligadas, denominada lista das adjacências. Esta representação assenta em três tipos definidos. Um tipo \emph{Vertex} que é um simples inteiro, uma estrutura \emph{Node} para ser usada nas listas ligadas, onde está contida a informação das capacidades e fluxos de cada arco e a terceira estrutura \emph{Graph} que contêm o vector das listas das adjacências e o seu tamanho. 

As principais funções do programa apresentado que merecem destaque são: 

\begin{lstlisting}
Graph GRAPHresidual(Graph G);
Vertex *bfs(Graph G, Vertex s, Vertex t, int *size, int *gray_size);
int ford_fulkerson(Graph G, Vertex s, Vertex t);
int main(int argc, char *argv[]); 
\end{lstlisting}


A função \emph{GRAPHresidual} recebe uma estrutura que representa o grafo que se pretende resolver, $G$, e retorna um nova estrutura que representa a sua rede residual, $G_f$. Internamente à função, são determinadas as capacidades residuais da rede residual a partir do fluxo e capacidade do grafo $G$ e filtram-se apenas as que são positivas. 

A função \emph{bfs} recebe um grafo $G$, um vértice de origem $s$ e um vértice de destino $t$ e devolve um vector de vértices que representam o caminho de $s$ até $t$. Os argumentos \emph{size} e \emph{gray\_size} são usados para retornar, via referências, o tamanho do vector do caminho e o número de arcos que a pesquisa não conseguiu visitar entre $s$ e $t$.

A lógica desta função consiste em percorrer todos vértices do grafo, marcando-os como não-visitados, em visita actual e visitados. Os vértices são percorridos recorrendo a uma fila, com o princípio FIFO (first in, first out), onde se vão colocando todos os vértices adjacentes do vértice em visita actual. A função termina quando atingir o vértice de destino $t$ ou quando a fila ficar vazia.


A função \emph{ford\_fulkerson} recebe a estrutura que representa o grafo $G$ e os vértices de origem e destino $s$ e $t$, respectivamente. O valor de retorno da função é o valor de corte mínimo do grafo, para o par origem-destino fornecido. 

A função começa por inicializar a zero o fluxo de todo o grafo e inicia um ciclo em que vai calcular a rede residual, determinar um caminho de aumento entre $s$ e $t$ na rede residual. Se o caminho existir, vai determinar a capacidade residual mínima ao longo desse caminho e vai aumentar o fluxo do grafo com o valor da capacidade residual calculado. Se o caminho não existir, vai retornar o valor \emph{gray\_size} que é devolvido pela função \emph{bfs}. Este valor corresponde ao número de arcos que a pesquisa não conseguiu transpor para atingir $t$ que equivale ao valor do corte mínimo e retorna-o.


A função \emph{main} é o início do programa, como todos em C, pelo que os seus argumentos de entrada bem como o valor de saída não necessitam de explicação. A função começa por ler do \emph{stdin} as linhas que correspondem ao grafo e constrói o objecto que o representa. 
Segue-se a leitura das $h$ linhas que contêm o conjunto dos pontos críticos a ser avaliados pelo programa e por cada uma determina-se o seu corte mínimo. 
O cálculo do corte mínimo de cada linha corresponde ao mínimo dos cortes da combinação $C_2^k$ dos $k$ pontos críticos, calculado com recurso à função \emph{ford\_fulkerson}, uma vez que a ordem dos pontos não afecta o cálculo e os pontos não se repetem.





\section*{Análise teórica}


Este programa executa em tempo linear ao tamanho do problema. A complexidade temporal de cada função do programa:

\begin{itemize}
  \item main $O(h n^2 V E^2)$
 \item leitura do input $\Theta(E)$
 \item  conjuntos de pontos críticos $O(h n^2 V E^2)$
   \item  determinar corte mínimo para conjunto de pontos críticos $O(n^2 V E^2)$ 
       \item Ford-Fulkerson $O(V E^2)$
           \item Rede Residual $O(E)$
          \item  BFS $O(V+E)$
\end{itemize}

em que $V$ são os vértices, $E$ são os arcos, $n$ o número de pontos críticos no conjunto e $h$ o número de conjuntos de pontos críticos. Assim, a complexidade final do programa é $O(h n^2 V E^2)$.



\section*{Resultados da avaliação experimental}

Esta solução passa com sucesso nos 16 testes presentes no sistema de testes automáticos Mooshak.

Foi verificado experimentalmente, que o tempo de execução varia com as variáveis definidas atrás, $V$, $E$, $n$ e $h$. Ou seja, quando se aumenta, por exemplo o número de pontos críticos de um conjunto, o tempo de execução aumentava, assim como para o número de conjuntos a determinar e o número de arcos e vértices. 



\begin{thebibliography}{9}

\bibitem{lamport94}
  Thomas H. Cormen,
  \emph{Introduction to Algorithms}.
  Charles E. Leiserson, Ronald L. Rivest and Clifford Stein,
  Third Edition,
  September 2009.
  
%  for .bib file
% @misc{bworld,
%  author = {Ingo Lütkebohle},
%  title = {{BWorld Robot Control Software}},
%  howpublished = "\url{http://aiweb.techfak.uni-bielefeld.de/content/bworld-robot-control-software/}",
%  year = {2008}, 
%  note = "[Online; accessed 19-July-2008]"
%}
 %http://en.wikipedia.org/wiki/Ford–Fulkerson_algorithm 
 %http://en.wikipedia.org/wiki/Edmonds–Karp_algorithm
 %http://community.topcoder.com/tc?module=Static&d1=tutorials&d2=maxFlow


\end{thebibliography}
\end{document}